\subsection{Zusammenhang von Graphen: Was ist das?}
\begin{frame}{Zusammenhang von Graphen}{Connectivity}
	\begin{block}{Streng zusammenhängender Graph}
		Ein streng zusammenhängender Graph ist ein gerichteter Graph,
		in dem jeder Knoten von jedem erreichbar ist.
	\end{block}
	\begin{figure}
		\begin{tikzpicture}[->,scale=1.8, auto,swap]
			% Draw a 7,11 network
			% First we draw the vertices
			\foreach \pos/\name in {{(0,0)/a}, {(0,2)/b}, {(1,2)/c},
				                    {(1,0)/d}, {(2,1)/e}, {(3,1)/f},
									{(3,2)/g}, {(2,0)/h}}
				\node[vertex] (\name) at \pos {$\name$};
			% Connect vertices with edges and draw weights
			\foreach \source/ \dest /\pos in {a/b/,b/c/,c/d/,d/a/,
										c/e/bend left, d/e/,e/c/, f/g/,
										g/f/bend left, d/h/}
				\path (\source) edge [\pos] node {} (\dest);
		\end{tikzpicture}
	\end{figure}
\end{frame}

\begin{frame}{Zusammenhang von Graphen}{Connectivity}
	\begin{block}{Zusammenhangskomponente}
		Eine Zusammenhangskomponente ist ein maximaler Subgraph S
		eines gerichteten Graphen, wobei S streng zusammenhängend ist.
		% Muss dieser Subgraph maximal sein?
	\end{block}
	\begin{figure}
		\begin{tikzpicture}[->,scale=1.8, auto,swap]
			% Draw a 7,11 network
			% First we draw the vertices
			\foreach \pos/\name in {{(0,0)/a}, {(0,2)/b}, {(1,2)/c},
				                    {(1,0)/d}, {(2,1)/e}, {(3,1)/f},
									{(3,2)/g}, {(2,0)/h}}
				\node[vertex] (\name) at \pos {$\name$};
			% Connect vertices with edges
			\foreach \source/ \dest /\pos in {a/b/,b/c/,c/d/,d/a/,
										c/e/bend left, d/e/,e/c/, f/g/,
										g/f/bend left, d/h/}
				\path (\source) edge [\pos] node {} (\dest);

			% colorize the vertices
			\foreach \vertex in {a,b,c,d,e}
				\path node[selected vertex] at (\vertex) {$\vertex$};

			\foreach \vertex in {f,g}
				\path node[blue vertex] at (\vertex) {$\vertex$};

			\foreach \vertex in {h}
				\path node[yellow vertex] at (\vertex) {$\vertex$};
		\end{tikzpicture}
	\end{figure}
\end{frame}

\begin{frame}{Elementare Eigenschaften}
	\begin{block}{}
		    Die Knotenmengen verschiedener SCCs sind disjunkt.
	\end{block}

	\begin{block}{}
		    SCCs bilden Zyklen.
	\end{block}

	\begin{block}{}
		    Die Vereinigung aller Knoten aller SCCs ergibt alle Knoten
			des ursprünglichen Graphen.
	\end{block}
\end{frame}
