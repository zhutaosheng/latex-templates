% A complete graph
% Author: Quintin Jean-Noël
% <http://moais.imag.fr/membres/jean-noel.quintin/>
\documentclass[varwidth=true, border=2pt]{standalone}

\usepackage{tikz}
\usetikzlibrary[topaths]


\begin{document}

% A counter, since TikZ is not clever enough (yet) to handle
% arbitrary angle systems.
\newcount\mycount

  \tikzstyle{vertexs}=[draw,fill=black,circle,minimum size=4pt,inner sep=0pt]

  \begin{tikzpicture}
      %the multiplication with floats is not possible. Thus I split the loop in two.
      \foreach \number in {1,...,8}{
          % Computer angle:
            \mycount=\number
            \advance\mycount by -1
      \multiply\mycount by 45
            \advance\mycount by 0
          \node[draw,circle,inner sep=0.25cm] (N-\number) at (\the\mycount:5.4cm) {};
        }
      \foreach \number in {9,...,16}{
          % Computer angle:
            \mycount=\number
            \advance\mycount by -1
      \multiply\mycount by 45
            \advance\mycount by 22.5
          \node[draw,circle,inner sep=0.25cm] (N-\number) at (\the\mycount:5.4cm) {};
        }
      \foreach \number in {1,...,15}{
            \mycount=\number
            \advance\mycount by 1
      \foreach \numbera in {\the\mycount,...,16}{
        \path (N-\number) edge[->,bend right=3] (N-\numbera)  edge[<-,bend
          left=3] (N-\numbera);
      }
    }
  \end{tikzpicture}
\end{document}
