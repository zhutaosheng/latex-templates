%!TEX root = Programmierparadigmen.tex
\chapter{Haskell}
\index{Haskell|(}
Haskell ist eine funktionale Programmiersprache, die 1990 in Version~1.0 veröffentlicht
wurde. Namensgeber ist Haskell Brooks Curry, der die mathematischen Grundlagen der funktionalen Programmierung entwickelte.

Wichtige Konzepte sind:
\begin{enumerate}
    \item Funktionen höherer Ordnung
    \item anonyme Funktionen (sog. Lambda-Funktionen)
    \item Pattern Matching
    \item Unterversorgung
    \item Typinferenz
\end{enumerate}

Haskell kann mit \enquote{Glasgow Haskell Compiler} mittels
\texttt{ghci} interpretiert und mittels

\section{Erste Schritte}
Haskell kann unter \href{http://www.haskell.org/platform/}{\path{www.haskell.org/platform/}}
für alle Plattformen heruntergeladen werden. Unter Debian-Systemen
ist das Paket \texttt{ghc} bzw. \texttt{haskell-platform} relevant.

\subsection{Hello World}
Speichere folgenden Quelltext als \texttt{hello-world.hs}:
\inputminted[linenos, numbersep=5pt, tabsize=4, frame=lines, label=hello-world.hs]{haskell}{scripts/haskell/hello-world.hs}

Kompiliere ihn mit \texttt{ghc -o hello hello-world.hs}. Es wird eine
ausführbare Datei erzeugt.

Alternativ kann es direkt mit \texttt{runghc hello-world.hs} ausgeführt werden.

\section{Syntax}
\subsection{Kommentare}\xindex{Kommentare!Haskell}
In Haskell werden kommentare durch \verb+--+ begonnen.

\subsection{Klammern und Funktionsdeklaration}
Haskell verzichtet an vielen Stellen auf Klammern. So werden im
Folgenden die Funktionen $f(x) := \frac{\sin x}{x}$ und $g(x) := x \cdot f(x^2)$
definiert:

\inputminted[numbersep=5pt, tabsize=4]{haskell}{scripts/haskell/einfaches-beispiel-klammern.hs}

Die Funktionsdeklarationen mit den Typen sind nicht notwendig, da
die Typen aus den benutzten Funktionen abgeleitet werden.

Zu lesen ist die Deklaration wie folgt:

\begin{center}
\texttt{[Funktionsname] :: \texttt{[Typendefinitionen]} => \texttt{Signatur}}
\end{center}

\begin{itemize}
    \item[T. Def.] Die Funktion \texttt{f} benutzt als Parameter bzw. Rückgabewert
          einen Typen. Diesen Typen nennen wir \texttt{a} und er ist
          vom Typ \texttt{Floating}. Auch \texttt{b}, \texttt{wasweisich}
          oder etwas ähnliches wäre ok.
    \item[Signatur] Die Signatur liest man am einfachsten von hinten:
        \begin{itemize}
            \item \texttt{f} bildet auf einen Wert vom Typ \texttt{a} ab und
            \item \texttt{f} hat genau einen Parameter \texttt{a}
        \end{itemize}
\end{itemize}

\todo[inline]{Gibt es Funktionsdeklarationen, die bis auf Wechsel des Namens und der Reihenfolge äquivalent sind?}

\subsection{if / else}\xindex{Haskell!if@\texttt{if}}%
Das folgende Beispiel definiert den Binomialkoeffizienten (vgl. \cref{bsp:binomialkoeffizient}):

\inputminted[numbersep=5pt, tabsize=4]{haskell}{scripts/haskell/binomialkoeffizient.hs}

Das könnte man auch mit sog. Guards machen:\xindex{Guard}

\inputminted[numbersep=5pt, tabsize=4]{haskell}{scripts/haskell/binomialkoeffizient-guard.hs}

\subsection{Rekursion}
Die Fakultätsfunktion wurde wie folgt implementiert:
\[fak(n) := \begin{cases}
        1              &\text{falls } n=0\\
        n \cdot fak(n) &\text{sonst}
    \end{cases}\]
\inputminted[numbersep=5pt, tabsize=4]{haskell}{scripts/haskell/fakultaet.hs}

Diese Implementierung benötigt $\mathcal{O}(n)$ rekursive Aufrufe und
hat einen Speicherverbrauch von $\mathcal{O}(n)$. Durch einen
\textbf{Akkumulator}\xindex{Akkumulator} kann dies verhindert werden:
\inputminted[numbersep=5pt, tabsize=4]{haskell}{scripts/haskell/fakultaet-akkumulator.hs}

\subsection{Listen}
Listen sind in Haskell 0-indiziert, d.~h. das erste Element jeder Liste hat
den Index 0.

\begin{itemize}
    \item \texttt{[]} erzeugt die leere Liste,
    \item \texttt{[1,2,3]} erzeugt eine Liste mit den Elementen $1, 2, 3$
    \item \texttt{:}\xindex{: (Haskell)} wird \textbf{cons}\xindex{cons} genannt und ist
          der Listenkonstruktor.
    \item \texttt{list !! i}\xindex{"!"! (Haskell)} gibt das $i$-te Element von \texttt{list} zurück.
    \item \texttt{head list} gibt den Kopf von \texttt{list} zurück,
          \texttt{tail list} den Rest:
          \inputminted[numbersep=5pt, tabsize=4]{haskell}{scripts/haskell/list-basic.sh}
    \item \texttt{last [1,9,1,3]} gibt 3 zurück.
    \item \texttt{length list} gibt die Anzahl der Elemente in \texttt{list} zurück.
    \item \texttt{maximum [1,9,1,3]} gibt 9 zurück (analog: \texttt{minimum}).
    \item \texttt{null list} prüft, ob \texttt{list} leer ist.
    \item \texttt{take 3 [1,2,3,4,5]} gibt \texttt{[1,2,3]} zurück.
    \item \texttt{drop 3 [1,2,3,4,5]} gibt \texttt{[4,5]} zurück.
    \item \texttt{reverse [1,9,1,3]} gibt \texttt{[3,1,9,1]} zurück.
    \item \texttt{elem item list} gibt zurück, ob sich \texttt{item} in \texttt{list} befindet.
\end{itemize}

\subsubsection{Beispiel in der interaktiven Konsole}
\inputminted[numbersep=5pt, tabsize=4]{haskell}{scripts/haskell/listenoperationen.sh}

\subsubsection{List-Comprehensions}\xindex{List-Comprehension}%
List-Comprehensions sind kurzschreibweisen für Listen, die sich an
der Mengenschreibweise in der Mathematik orientieren. So entspricht
die Menge
\begin{align*}
    myList &= \Set{1,2,3,4,5,6}\\
    test   &= \Set{x \in myList | x > 2}
\end{align*}
in etwa folgendem Haskell-Code:
\inputminted[numbersep=5pt, tabsize=4]{haskell}{scripts/haskell/list-comprehensions.sh}

\begin{beispiel}[List-Comprehension]
    Das folgende Beispiel zeigt, wie man mit List-Comprehensions die unendliche
    Liste aller pythagoreischen Tripels erstellen kann:

    \inputminted[numbersep=5pt, tabsize=4]{haskell}{scripts/haskell/pythagorean-triples.hs}
\end{beispiel}

\begin{beispiel}[List-Comprehension]
    Das folgende Beispiel zeigt, wie man die unendlich große Menge

    \[\Set{p^i | p \text{ Primzahl}, 1 \leq i \leq n}\]

    erstellt:

    \inputminted[numbersep=5pt, tabsize=4]{haskell}{scripts/haskell/primepowers.hs}

    Dabei ist die Reihenfolge wichtig. Würde man zuerst die \verb+i+ und dann
    die Primzahlen notieren, würde Haskell versuchen erst alle Primzahlen durch
    zu gehen. Da dies eine unendliche Liste ist, würde also niemals die zweite
    Potenz einer Primzahl auftauchen.
\end{beispiel}

\subsection{Strings}
\begin{itemize}
    \item Strings sind Listen von Zeichen:\\
          \texttt{tail "ABCDEF"} gibt \texttt{"BCDEF"} zurück.
\end{itemize}

\subsection{Let und where}\xindex{let (Haskell)@\texttt{let} (Haskell)}\xindex{where (Haskell)@\texttt{where} (Haskell)}%
\inputminted[numbersep=5pt, tabsize=4]{haskell}{scripts/haskell/let-where-bindung.hs}

\subsection{Funktionskomposition}\xindex{. (Haskell)}\xindex{Funktionskomposition}%
In Haskell funktioniert Funktionskomposition mit einem Punkt:

\inputminted[numbersep=5pt, tabsize=4]{haskell}{scripts/haskell/function-composition.hs}

Dabei ergibt \texttt{h (-3)} in der mathematischen Notation
\[(g \circ f) (-3) = f(g(-3)) = f(-4) = 16\]
und \texttt{i (-3)} ergibt
\[(f \circ g) (-3) = g(f(-3)) = g(9) = 8\]
Es ist also anzumerken, dass die Reihenfolge der mathematischen Konvention entspricht.

\subsection{\$ (Dollar-Zeichen) und ++}\xindex{\$ (Haskell)}\xindex{++ (Haskell)@\texttt{++} (Haskell)}%
Das Dollar-Zeichen \$ dient in Haskell dazu Klammern zu vermeiden. So sind die
folgenden Zeilen äquivalent:

\inputminted[numbersep=5pt, tabsize=4]{haskell}{scripts/haskell/dollar-example.hs}

Das doppelte Plus (\texttt{++}) wird verwendet um Listen mit einander zu verbinden.

\subsection{Logische Operatoren}\xindex{Haskell!Logische Operatoren}

\begin{table}[H]
    \centering
    \begin{tabular}{CCCC}
    UND    & ODER     & Wahr  & Falsch \\ \hline\hline
    \&\&   & ||       & True  & False \\[4ex]
    GLEICH & UNGLEICH & NICHT & ~     \\ \hline\hline
    ==     & /=       & not   & ~     \\
    \end{tabular}
    \caption{Logische Operatoren in Haskell}\xindex{Logische Operatoren!Haskell}
\end{table}

\section{Typen}
\subsection{Standard-Typen}
Haskell kennt einige Basis-Typen:
\begin{itemize}
  \item \textbf{Int}: Ganze Zahlen. Der Zahlenbereich kann je nach Implementierung variieren,
  aber der Haskell-Standart garantiert, dass das Intervall
  $[-2^{29}, 2^{29}-1]$ abgedeckt wird.
  \item \textbf{Integer}: beliebig große ganze Zahlen
  \item \textbf{Float}: Fließkommazahlen
  \item \textbf{Double}: Fließkommazahlen mit doppelter Präzision
  \item \textbf{Bool}: Wahrheitswerte
  \item \textbf{Char}: Unicode-Zeichen
\end{itemize}

Des weiteren gibt es einige strukturierte Typen:
\begin{itemize}
  \item Listen: z.~B. \texttt{[1,2,3]}
  \item Tupel: z.~B. \texttt{(1,'a',2)}
  \item Brüche (Fractional, RealFrac)
  \item Summen-Typen: Typen mit mehreren möglichen Repräsentationen
\end{itemize}

\subsection{Typinferenz}
In Haskell werden Typen aus den Operationen geschlossfolgert. Dieses
Schlussfolgern der Typen, die nicht explizit angegeben werden müssen,
nennt man \textbf{Typinferent}\xindex{Typinferenz}.


Haskell kennt die Typen aus \cref{fig:haskell-type-hierarchy}.

Ein paar Beispiele zur Typinferenz:
\inputminted[numbersep=5pt, tabsize=4]{haskell}{scripts/haskell/typinferenz.sh}

\begin{figure}[htp]
    \centering
    \resizebox{0.9\linewidth}{!}{\begin{tikzpicture}
    \tikzstyle{node}=[ellipse,thick,fill=white,draw=black,inner sep=0pt,text width=3cm,align=center]
    \tikzstyle{edge}=[->, ultra thick]
    \matrix[row sep=0.5cm,column sep=0.5cm] {
    \node[node] (Eq)         {\textbf{Eq}\\All except IO, (->)};   &
    \node[node] (Show)       {\textbf{Show}\\All except IO, (->)}; &
    \node[node] (Read)       {\textbf{Read}\\All except IO, (->)}; \\
    \node[node] (Ord)        {\textbf{Ord}\\All except IO, (->), IOError};   &
    \node[node] (Num)        {\textbf{Num}\\Int, Integer, Float, Double}; &
    \node[node] (Bounded)    {\textbf{Bounded}\\Int, Char, Bool, (), Ordering, tuples}; \\
    \node[node] (Enum)       {\textbf{Enum}\\{\small (), Bool, Char, Ordering, Int, Integer, Float, Double}};   &
    \node[node] (Real)       {\textbf{Real}\\Int, Integer, Float, Double}; &
    \node[node] (Fractional) {\textbf{Fractional}\\Float, Double}; \\
    \node[node] (Integral)   {\textbf{Integral}\\Int, Integer};   &
    \node[node] (RealFrac)   {\textbf{RealFrac}\\Float, Double}; &
    \node[node] (Floating)   {\textbf{Floating}\\Float, Double}; \\
    \node[node] (Monad)      {\textbf{Monad}\\IO, (), Maybe};   &
    \node[node] (RealFloat)  {\textbf{RealFloat}\\Float, Double}; &
     \\
    \node[node] (MonadPlus)  {\textbf{MonadPlus}\\IO, (), Maybe};   &
    \node[node] (Functor)    {\textbf{Functor}\\IO, (), Maybe}; &
     \\
    };
    \draw[edge] (Eq) -- (Ord);
    \draw[edge] (Eq) -- (Num);
    \draw[edge] (Show) -- (Num);
    \draw[edge] (Ord) -- (Real);
    \draw[edge] (Num) -- (Real);
    \draw[edge] (Num) -- (Fractional);
    \draw[edge] (Enum) -- (Integral);
    \draw[edge] (Real) -- (Integral);
    \draw[edge] (Real) -- (RealFrac);
    \draw[edge] (Floating) -- (RealFloat);
    \draw[edge] (RealFrac) -- (RealFloat);
    \draw[edge] (Monad) -- (MonadPlus);
\end{tikzpicture}
}
    \caption{Hierarchie der Haskell Standardklassen}
    \label{fig:haskell-type-hierarchy}
\end{figure}

\subsection{type}\xindex{type (Haskell)@\texttt{type} (Haskell)}%
Mit \texttt{type} können Typsynonyme erstellt werden:

\inputminted[numbersep=5pt, tabsize=4]{haskell}{scripts/haskell/alt-types.hs}

\subsection{data}\xindex{data (Haskell)@\texttt{data} (Haskell)}%
Mit dem Schlüsselwort \texttt{data} können algebraische Datentypen\xindex{Datentyp!algebraischer}
erzeugt werden:

\inputminted[numbersep=5pt, tabsize=4]{haskell}{scripts/haskell/data-example.hs}

\section{Lazy Evaluation}\xindex{Lazy Evaluation}%
Haskell wertet Ausdrücke nur aus, wenn es nötig ist.

\begin{beispiel}[Lazy Evaluation]
    Obwohl der folgende Ausdruck einen Teilausdruck hat, der einen Fehler zurückgeben
    würde, kann er aufgrund der Lazy Evaluation zu 2 evaluiert werden:
    \inputminted[linenos, numbersep=5pt, tabsize=4, frame=lines, label=lazy-evaluation.hs]{haskell}{scripts/haskell/lazy-evaluation.hs}
\end{beispiel}

\section{Beispiele}

\subsection{Primzahlsieb}\xindex{Primzahlsieb}%
\inputminted[linenos, numbersep=5pt, tabsize=4, frame=lines, label=primzahlsieb.hs]{haskell}{scripts/haskell/primzahlsieb.hs}

\subsection{Quicksort}\xindex{filter (Haskell)@\texttt{filter} (Haskell)}%
\inputminted[linenos, numbersep=5pt, tabsize=4, frame=lines, label=qsort.hs]{haskell}{scripts/haskell/qsort.hs}

\begin{itemize}
    \item Die leere Liste ergibt sortiert die leere Liste.
    \item Wähle das erste Element \texttt{p} als Pivotelement und
          teile die restliche Liste \texttt{ps} in kleinere und
          gleiche sowie in größere Elemente mit \texttt{filter} auf.
          Konkateniere diese beiden Listen mit \texttt{++}.
\end{itemize}

Durch das Ausnutzen von Unterversorgung\xindex{Unterversorgung} lässt
sich das ganze sogar noch kürzer schreiben:

\inputminted[linenos, numbersep=5pt, tabsize=4, frame=lines, label=qsort.hs]{haskell}{scripts/haskell/qsort-unterversorg.hs}

\subsection{Fibonacci}\xindex{Fibonacci}
\inputminted[linenos, numbersep=5pt, tabsize=4, frame=lines, label=fibonacci.hs]{haskell}{scripts/haskell/fibonacci.hs}
\inputminted[linenos, numbersep=5pt, tabsize=4, frame=lines, label=fibonacci-akk.hs]{haskell}{scripts/haskell/fibonacci-akk.hs}
\xindex{zipWith (Haskell)@\texttt{zipWith} (Haskell)}
\inputminted[linenos, numbersep=5pt, tabsize=4, frame=lines, label=fibonacci-zip.hs]{haskell}{scripts/haskell/fibonacci-zip.hs}
\inputminted[linenos, numbersep=5pt, tabsize=4, frame=lines, label=fibonacci-pattern-matching.hs]{haskell}{scripts/haskell/fibonacci-pattern-matching.hs}

Die unendliche Liste alle Fibonacci-Zahlen, also der Fibonacci-\textit{Stream}\xindex{Stream}
wird wie folgt erzeugt:

\inputminted[linenos, numbersep=5pt, tabsize=4, frame=lines, label=fibonacci-stream.hs]{haskell}{scripts/haskell/fibonacci-stream.hs}

\subsection{Polynome}\xindex{Polynome}\xindex{zipWith (Haskell)@\texttt{zipWith} (Haskell)}\xindex{Folds!foldr (Haskell)@\texttt{foldr} (Haskell)}\xindex{tail (Haskell)@\texttt{tail} (Haskell)}%
\inputminted[linenos, numbersep=5pt, tabsize=4, frame=lines, label=polynome.hs]{haskell}{scripts/haskell/polynome.hs}

\subsection{Hirsch-Index}\xindex{Hirsch-Index}\xindex{Ord}\xindex{Num}%
\textbf{Parameter}: Eine Liste $L$ von Zahlen aus $\mdn$\\
\textbf{Rückgabe}: $\max \Set{n \in \mdn | n \leq \|\Set{i \in L | i \geq n}\|}$

\inputminted[linenos, numbersep=5pt, tabsize=4, frame=lines, label=hirsch-index.hs]{haskell}{scripts/haskell/hirsch-index.hs}

\subsection{Lauflängencodierung}\xindex{Lauflängencodierung}\xindex{splitWhen}\xindex{group}\xindex{concat}%

\inputminted[linenos, numbersep=5pt, tabsize=4, frame=lines, label=lauflaengencodierung.hs]{haskell}{scripts/haskell/lauflaengencodierung.hs}

\subsection{Intersections}\xindex{Intersections}\xindex{List-Comprehension}%

\inputminted[linenos, numbersep=5pt, tabsize=4, frame=lines, label=intersect.hs]{haskell}{scripts/haskell/intersect.hs}

\subsection{Funktionen höherer Ordnung}\xindex{Folds}\xindex{Folds!foldl (Haskell)@\texttt{foldl} (Haskell)}\xindex{Folds!foldr (Haskell)@\texttt{foldr} (Haskell)}\label{bsp:foldl-und-foldr}
\inputminted[linenos, numbersep=5pt, tabsize=4, frame=lines, label=folds.hs]{haskell}{scripts/haskell/folds.hs}

\subsection{Chruch-Zahlen}
\inputminted[linenos, numbersep=5pt, tabsize=4, frame=lines, label=church.hs]{haskell}{scripts/haskell/church.hs}

\subsection{Trees}\xindex{tree}\xindex{map!tree}%
Einen Binärbaum kann man in Haskell so definieren:

\inputminted[numbersep=5pt, tabsize=4]{haskell}{scripts/haskell/binary-tree.hs}

Einen allgemeinen Baum so:

\inputminted[numbersep=5pt, tabsize=4]{haskell}{scripts/haskell/general-tree.hs}

Hier ist \texttt{t} der polymorphe Typ des Baumes. \texttt{t} gibt also an welche
Elemente der Baum enthält.

Man kann auf einem solchen Baum auch eine Variante von \texttt{map} und
\texttt{reduce} definieren,
also eine Funktion \texttt{mapT}, die eine weitere Funktion \texttt{f} auf jeden
Knoten anwendet:

\inputminted[numbersep=5pt, tabsize=4]{haskell}{scripts/haskell/mapt.hs}

\subsection{Standard Prelude}
Hier sind die Definitionen eininger wichtiger Funktionen:
\xindex{zipWith (Haskell)@\texttt{zipWith} (Haskell)}\xindex{zip (Haskell)@\texttt{zip} (Haskell)}
\inputminted[numbersep=5pt, tabsize=4]{haskell}{scripts/haskell/standard-definitions.hs}

\section{Weitere Informationen}
\begin{itemize}
    \item \href{http://hackage.haskell.org/package/base-4.6.0.1}{\path{hackage.haskell.org/package/base-4.6.0.1}}: Referenz
    \item \href{http://www.haskell.org/hoogle/}{\path{haskell.org/hoogle}}: Suchmaschine für das Haskell-Manual
    \item \href{http://wiki.ubuntuusers.de/Haskell}{\path{wiki.ubuntuusers.de/Haskell}}: Hinweise zur Installation von Haskell unter Ubuntu
    \item \href{http://learnyouahaskell.com/chapters}{learnyouahaskell.com/chapters}
\end{itemize}

\index{Haskell|)}

