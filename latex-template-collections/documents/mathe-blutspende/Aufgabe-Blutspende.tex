\documentclass[a4paper,9pt]{scrartcl}
\usepackage[ngerman]{babel}
\usepackage[utf8]{inputenc}
\usepackage{amssymb,amsmath}
\usepackage{geometry}
\usepackage{graphicx}

\geometry{a4paper,left=18mm,right=18mm, top=1cm, bottom=2cm}

\setcounter{secnumdepth}{2}
\setcounter{tocdepth}{2}

\begin{document}
 \title{Blutabnahme}
 \author{Martin Thoma}

 \setcounter{section}{1}
 \section*{Aufgabenstellung}
    Ein Mensch hat ca. 5 Liter Blut. Bei einer Blutspende wird in der Regel etwa
    ein halber Liter Blut entnommen. Bis zur nächsten Blutspende ist wird dieses
    Blut wieder neu gebildet. Wie häufig muss Blut gespendet werden, bis 95\%
    des ursprünglichen Blutes gespendet wurde?\\

    \noindent Die natürliche Neubildung von Blut auch ohne Blutspende wird vernachlässigt.

    \subsection{Die ersten Werte}
    $f(x)$ sei die Menge des ursprünglichen Blutes, das nach $x$ Spenden gespendet
    wurde:\\
    $f(0) = 0$\\
    Beim ersten mal Blutspenden wird ein halber Liter des ursprünglichen Blutes
    gespendet:\\
    $f(1) = f(0) + 0{,}5$\\
    Beim zweiten mal Blutspenden werden 0,45 Liter des ursprünglichen Blutes
    gespendet:\\
    $f(2) = f(1) + f(0) + \frac{5-0{,}5}{5} \cdot 0{,}5 Liter = 0{,}95 Liter $\\
    Beim dritten mal Blutspenden werden 0,405 Liter des ursprünglichen Blutes
    gespendet:\\
    $f(3) = f(2) + f(1) + f(0) + \frac{5-0{,}95}{5} \cdot 0{,}5 Liter = 1{,}355 Liter$

    \subsection{Eine rekursive Formel}
        \begin{align}
            f(1) &= 0{,}5 \\
            f(x) &= \frac{5-f(x-1)}{5} \cdot 0{,}5 + f(x-1)
        \end{align}

    \subsection{Auflösen der Rekursion}
        \begin{align}
             f(3) &= 0{,}5 + \frac{9}{10} \cdot (0{,}5 + \frac{9}{10} \cdot (0{,}5 + \frac{9}{10} \cdot 0{,}5))\\
                  &= 0{,}5 + \frac{9}{10} \cdot 0{,}5 + (\frac{9}{10})^2 \cdot (0{,}5 + \frac{9}{10} \cdot 0{,}5)\\
                  &= 0{,}5 + \frac{9}{10} \cdot 0{,}5 + (\frac{9}{10})^2 \cdot 0{,}5 + (\frac{9}{10})^3 \cdot 0{,}5\\
                  &= 0{,}5 \cdot (1 + \frac{9}{10} + (\frac{9}{10})^2 + (\frac{9}{10})^3 \cdot )\\
              f(x)&= \frac{1}{2} \cdot \sum_{i=0}^{x} (\frac{9}{10})^i
        \end{align}

    \subsection{Auflösen des Summensymbols}
        \begin{align}
            f(x) &= \frac{1}{2} \cdot \sum_{i=0}^{x} (\frac{9}{10})^i\\
                 &= \frac{1}{2}\cdot (\frac{0{,}9^{x+1} - 1}{0{,}9 - 1})\\
                 &= \frac{1}{2}\cdot (-10 \cdot 0{,}9^{x+1} + 10)\\
                 &= -5 \cdot 0{,}9^{x+1} + 5\\
                 &= 5 \cdot (1 - 0{,}9^{x+1})
        \end{align}

    \subsection{Lösung}
        \begin{align}
            0{,}95 \cdot 5 &= 5 \cdot (1- 0{,}9^{x+1})\\
                    0{,}95 &= 1 - 0{,}9^{x+1}\\
              0{,}9^{x+1} &= 0{,}05\\
    \ln(0{,}9) \cdot {x+1} &= \ln(0{,}05) \\
                     x  &= \frac{\ln(0{,}05)}{\ln(0{,}9)} - 1\\
                     x  &= 27{,}43
        \end{align}
    \subsection{Antwort}
        Nach dem 28. mal Blutspenden wurden 95\% des ursprünglichen Blutes
        gespendet.
\end{document}
