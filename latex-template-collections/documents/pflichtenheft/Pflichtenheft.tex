\documentclass[a4paper,12pt]{article}
\usepackage{amssymb} % needed for math
\usepackage{amsmath} % needed for math
\usepackage[utf8]{inputenc} % this is needed for german umlauts
\usepackage[ngerman]{babel} % this is needed for german umlauts
\usepackage[T1]{fontenc}    % this is needed for correct output of umlauts in pdf
\usepackage[margin=2.5cm]{geometry} %layout
\usepackage{booktabs}

% this is needed for forms and links within the text
\usepackage{hyperref}

% glossar, see http://en.wikibooks.org/wiki/LaTeX/Glossary
% has to be loaded AFTER hyperref so that entries are clickable
\usepackage[nonumberlist]{glossaries}

% The following is needed in order to make the code compatible
% with both latex/dvips and pdflatex.
\ifx\pdftexversion\undefined
\usepackage[dvips]{graphicx}
\else
\usepackage[pdftex]{graphicx}
\DeclareGraphicsRule{*}{mps}{*}{}
\fi

\makeglossary

%%%%%%%%%%%%%%%%%%%%%%%%%%%%%%%%%%%%%%%%%%%%%%%%%%%%%%%%%%%%%%%%%%%%%%
% Variablen                                 						 %
%%%%%%%%%%%%%%%%%%%%%%%%%%%%%%%%%%%%%%%%%%%%%%%%%%%%%%%%%%%%%%%%%%%%%%
\newcommand{\authorName}{Martin Thoma}
\newcommand{\auftraggeber}{Karlsruhe Institute of Technology}
\newcommand{\auftragnehmer}{Eine Firma}
\newcommand{\projektName}{UpToDaTe}
\newcommand{\tags}{\authorName, Pflichtenheft, KIT, Informatik, PSE}
\newcommand{\glossarName}{Glossar}
\title{\projektName~(Pflichtenheft)}
\author{\authorName}
\date{\today}

%%%%%%%%%%%%%%%%%%%%%%%%%%%%%%%%%%%%%%%%%%%%%%%%%%%%%%%%%%%%%%%%%%%%%%
% PDF Meta information                                 				 %
%%%%%%%%%%%%%%%%%%%%%%%%%%%%%%%%%%%%%%%%%%%%%%%%%%%%%%%%%%%%%%%%%%%%%%
\hypersetup{
  pdfauthor   = {\authorName},
  pdfkeywords = {\tags},
  pdftitle    = {\projektName~(Pflichtenheft)}
}

%%%%%%%%%%%%%%%%%%%%%%%%%%%%%%%%%%%%%%%%%%%%%%%%%%%%%%%%%%%%%%%%%%%%%%
% Create a shorter version for tables. DO NOT CHANGE               	 %
%%%%%%%%%%%%%%%%%%%%%%%%%%%%%%%%%%%%%%%%%%%%%%%%%%%%%%%%%%%%%%%%%%%%%%
\newcommand\addrow[2]{#1 &#2\\ }

\newcommand\addheading[2]{#1 &#2\\ \hline}
\newcommand\tabularhead{\begin{tabular}{lp{13cm}}
\hline
}

\newcommand\addmulrow[2]{ \begin{minipage}[t][][t]{2.5cm}#1\end{minipage}%
   &\begin{minipage}[t][][t]{8cm}
    \begin{enumerate} #2   \end{enumerate}
    \end{minipage}\\ }

\newenvironment{usecase}{\tabularhead}
{\hline\end{tabular}}

\usepackage{microtype}


%%%%%%%%%%%%%%%%%%%%%%%%%%%%%%%%%%%%%%%%%%%%%%%%%%%%%%%%%%%%%%%%%%%%%%
% THE DOCUMENT BEGINS             	                              	 %
%%%%%%%%%%%%%%%%%%%%%%%%%%%%%%%%%%%%%%%%%%%%%%%%%%%%%%%%%%%%%%%%%%%%%%
\begin{document}
 \pagenumbering{roman}
 \begin{titlepage}
\maketitle
\thispagestyle{empty} % no page number

\begin{verbatim}












\end{verbatim}


  \begin{tabular}[t]{ll}
	Projekt:       & \quad \projektName \\[1.2ex]
	Auftraggeber:  & \quad \auftraggeber\\[1.2ex]
	Auftragnehmer: & \quad \auftragnehmer\\[1.2ex]
  \end{tabular}

\begin{tabular}{|p{3 cm}|p{3 cm}|p{5 cm}|}
\hline
\textbf{Version} & \textbf{Datum} & \textbf{Autor(en)} \\
\hline
\hline
1.0 & 26.04.2012 & \authorName \\
\hline
\end{tabular}
\end{titlepage}
         % Deckblatt.tex laden und einfügen
 \setcounter{page}{2}
 \tableofcontents          % Inhaltsverzeichnis ausgeben
 \clearpage
 \pagenumbering{arabic}

\section{Zielbestimmung}
\subsection{Musskriterien}
\subsection{Wunschkriterien}
\subsection{Abgrenzungskriterien}
% Was will ich bewusst nicht umsetzen?
% Was soll es nicht sein?

\section{Produkteinsatz}
% Zielgruppe
% Anwendungsbereiche
% Betriebsbedinugen
% Wer? Was? Wozu?

\section{Produktumgebung}
% Unter welcher Software / Hardware läuft es?

\section{Funktionale Anforderungen}
% Was soll das Produkt machen können

\section{Produktdaten}
% Was soll gespeichert werden?

\subsection{Personendaten}
\subsection{Messdaten}

\section{Nichtfunktionale Anforderungen}

\section{Benutzeroberfläche}

\section{Qualitätszielbestimmungen}
% Tabelle ... schafeln .. was ist ihm wichtig

\section{Globale Testfälle und Szenarien}
\subsection{Globale Testfälle}
\subsubsection{Grundlegende Testfälle}
\subsubsection{Erweiterte Testfälle}
\subsection{Szenarien}
\section{Entwicklungsumgebung}

\clearpage
%%%%%%%%%%%%%%%%%%%%%%%%%%%%%%%%%%%%%%%%%%%%%%%%%%%%%%%%%%%%%%%%%%%%%%
% Begriffslexikon zur Beschreibung des Produkts						 %
%%%%%%%%%%%%%%%%%%%%%%%%%%%%%%%%%%%%%%%%%%%%%%%%%%%%%%%%%%%%%%%%%%%%%%
%\newglossaryentry{sortierschluessel}
%{
%  name=Sortierschlüssel,
%  description={ein Schlüssel, anhand dessen diese Einträge sortiert werden}
%}
\newglossaryentry{uptodate}
{
  name=UpToDaTe,
  description={Upload Automation and Data Entry}
}
\newglossaryentry{nir}
{
  name=NIR,
  description={Nahes Infrarot}
}
%\newacronym{abc}{Blub}{Bananarama}

% Setze den richtigen Namen für das Glossar
\renewcommand*{\glossaryname}{\section{\glossarName}}

% Drucke das gesamte Glossar
\glsaddall
\printglossaries

% Trage das Glossar in das Inhaltsverzeichnis ein
\stepcounter{section}
\addcontentsline{toc}{section}{\numberline {\thesection} \glossarName}

\end{document}
