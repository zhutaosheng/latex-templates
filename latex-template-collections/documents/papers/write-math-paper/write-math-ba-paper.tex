\documentclass[9pt,technote,a4paper]{IEEEtran}
\usepackage{amssymb, amsmath} % needed for math

\usepackage[a-1b]{pdfx}
\usepackage{filecontents}
\begin{filecontents*}{\jobname.xmpdata}
    \Keywords{recognition\sep machine learning\sep neural networks\sep symbols\sep multilayer perceptron}
    \Title{On-line Recognition of Handwritten Mathematical Symbols}
    \Author{Martin Thoma, Kevin Kilgour, Sebastian St{\"u}ker and Alexander Waibel}
    \Org{Institute for Anthropomatics and Robotics}
    \Doi{}
\end{filecontents*}

\RequirePackage{ifpdf}
\ifpdf \PassOptionsToPackage{pdfpagelabels}{hyperref} \fi
\RequirePackage{hyperref}
\usepackage{parskip}
\usepackage[pdftex,final]{graphicx}
\usepackage{csquotes}
\usepackage{braket}
\usepackage{booktabs}
\usepackage{multirow}
\usepackage{pgfplots}
\usepackage{wasysym}
\usepackage{caption}
% \captionsetup{belowskip=12pt,aboveskip=4pt}
\makeatletter
\newcommand\mynobreakpar{\par\nobreak\@afterheading}
\makeatother
\usepackage[noadjust]{cite}
\usepackage[nameinlink,noabbrev]{cleveref} % has to be after hyperref, ntheorem, amsthm
\usepackage[binary-units,group-separator={,}]{siunitx}
\sisetup{per-mode=fraction,binary-units=true}
\DeclareSIUnit\pixel{px}
\usepackage{glossaries}
\loadglsentries[main]{glossary}
\makeglossaries

\title{On-line Recognition of Handwritten Mathematical Symbols}
\author{Martin Thoma, Kevin Kilgour, Sebastian St{\"u}ker and Alexander Waibel}

\hypersetup{
  pdfauthor   = {Martin Thoma\sep Kevin Kilgour\sep Sebastian St{\"u}ker\sep Alexander Waibel},
  pdfkeywords = {recognition\sep machine learning\sep neural networks\sep symbols\sep multilayer perceptron},
  pdfsubject  = {Recognition},
  pdftitle    = {On-line Recognition of Handwritten Mathematical Symbols},
}
\newcommand{\totalCollectedRecordings}{166898}  % ACTUALITY
\newcommand{\detexifyCollectedRecordings}{153423}
\newcommand{\trainingsetsize}{134804}
\newcommand{\validtionsetsize}{15161}
\newcommand{\testsetsize}{17012}
\newcommand{\totalClasses}{1111}
\newcommand{\totalClassesAnalyzed}{369}
\newcommand{\totalClassesAboveFifty}{680}
\newcommand{\totalClassesNotAnalyzedBelowFifty}{431}
\newcommand{\detexifyPercentage}{$\SI{91.93}{\percent}$}
\newcommand{\recordingsWithDots}{$\SI{2.77}{\percent}$}  % excluding i,j, ...
\newcommand{\recordingsWithDotsSizechange}{$\SI{0.85}{\percent}$}  % excluding i,j, ...
\crefname{table}{Table}{Tables}
\crefname{figure}{Figure}{Figures}

%%%%%%%%%%%%%%%%%%%%%%%%%%%%%%%%%%%%%%%%%%%%%%%%%%%%%%%%%%%%%%%%%%%%%
% Begin document                                                    %
%%%%%%%%%%%%%%%%%%%%%%%%%%%%%%%%%%%%%%%%%%%%%%%%%%%%%%%%%%%%%%%%%%%%%
\begin{document}
\maketitle
\begin{abstract}
The automatic recognition of single handwritten symbols has three main
applications: Supporting users who know how a symbol looks like, but not what
its name is, providing the necessary commands for professional publishing, or
as a building block for formula recognition.

This paper presents a system which uses the pen trajectory to classify
handwritten symbols. Five preprocessing steps, one data multiplication
algorithm, five features and five variants for multilayer Perceptron training
were evaluated using $\num{166898}$ recordings. Those recordings were made
publicly available. The evaluation results of these 21~experiments were used to
create an optimized recognizer which has a top-1 error of less than
$\SI{17.5}{\percent}$ and a top-3 error of $\SI{4.0}{\percent}$. This is a
relative improvement of $\SI{18.5}{\percent}$ for the top-1 error and
$\SI{29.7}{\percent}$ for the top-3 error compared to the baseline system. This
improvement was achieved by \acrlong{SLP} and adding new features. The
improved classifier can be used via \href{http://write-math.com/}{write-math.com}.
\end{abstract}

%!TEX root = pdfa-paper.tex

\section{Introduction}
Lorem ipsum dolor sit amet, consectetur adipiscing elit. Aenean id dolor a eros tempor accumsan. Ut ante dui, lacinia et interdum quis, suscipit eu lacus. Pellentesque eget tellus arcu. Vestibulum et nunc at arcu viverra ultrices. Cum sociis natoque penatibus et magnis dis parturient montes, nascetur ridiculus mus. Phasellus blandit pharetra lacus, sit amet luctus quam aliquam sit amet.
\[e^{i \pi}= -1\]
Nam et lacus luctus tellus feugiat rhoncus at eu enim. Sed a eros a nibh suscipit scelerisque. Cum sociis natoque penatibus et magnis dis parturient montes, nascetur ridiculus mus. Donec erat massa, ullamcorper condimentum justo quis, congue ultrices elit. Morbi efficitur nec elit non accumsan. In hendrerit, leo eu fermentum posuere, risus quam porttitor massa, non eleifend lorem metus faucibus nulla. Nunc lacinia tellus et purus accumsan consequat.

\begin{figure}[htb]
    \centering
    \begin{tikzpicture}
    \begin{axis}[
        legend pos=south west,
        axis x line=middle,
        axis y line=middle,
        grid = major,
        grid style={dashed, gray!30},
        xmin=-1,        % start the diagram at this x-coordinate
        xmax= 6,        % end   the diagram at this x-coordinate
        ymin=-0.25,     % start the diagram at this y-coordinate
        ymax= 2.25,     % end   the diagram at this y-coordinate
        axis background/.style={fill=white},
        xlabel=$x$,
        ylabel=$y$,
        tick align=outside,
        minor tick num=-3,
        enlargelimits=true,
        tension=0.08]
      \addplot[domain=0:1, red, thick,samples=20] {0.5*x*x};
      \addplot[domain=1:2, green, thick,samples=20] {x-0.5};
      \addplot[domain=2:3, blue, thick,samples=500] {-0.5*(x-2)*(x-2)+x-0.5};
      \addplot[domain=3:5, purple, thick,samples=20] {5-x};
      \addplot[domain=5:7, orange, thick,samples=3] {0};
      \addplot[domain=-3:0, orange, thick,samples=3] {0};
    \end{axis}
\end{tikzpicture}
    \caption{Training- and test error by number of trained epochs for different
             topologies with. The plot shows
             that all pretrained systems performed much better than the systems
             without pretraining. All plotted systems did not improve
             with more epochs of training.}
\label{fig:training-and-test-error-for-different-topologies-pretraining}
\end{figure}

Suspendisse nec scelerisque lectus. Nam id arcu sapien. Phasellus id pharetra metus, et ultricies elit. Proin ultricies sodales nibh, ac ultricies arcu. Nullam lacinia urna tempus lectus tempus porttitor. Praesent malesuada dapibus purus id aliquet. Cras eleifend vestibulum nunc, nec luctus augue dapibus id. Duis dapibus massa id luctus commodo. Sed ornare erat vitae condimentum commodo. Integer bibendum odio id auctor condimentum. Praesent aliquet justo nec lacus convallis sagittis. Aliquam eget orci lacus. Vestibulum orci tortor, posuere a dui nec, ultrices elementum orci. Phasellus felis eros, iaculis dictum ipsum eget, bibendum placerat arcu.

Sed vel purus elit. Aenean ornare ex vitae lectus malesuada, id blandit nisl tempor. Donec pulvinar interdum risus a fermentum. Mauris luctus tempor nunc. Cras viverra semper libero, in lobortis sapien ultrices lobortis. Suspendisse ac tempus sem. Suspendisse potenti. Suspendisse malesuada porta ligula eget maximus. Fusce facilisis accumsan diam, vel blandit turpis placerat in.
%!TEX root = write-math-ba-paper.tex

\section{General System Design}
The following steps are used for symbol classification:\nobreak
\begin{enumerate}
    \item \textbf{Preprocessing}: Recorded data is never perfect. Devices have
          errors and people make mistakes while using the devices. To tackle
          these problems there are preprocessing algorithms to clean the data.
          The preprocessing algorithms can also remove unnecessary variations
          of the data that do not help in the classification process, but hide
          what is important. Having slightly different sizes of the same symbol
          is an example of such a variation. Four preprocessing algorithms that
          clean or normalize recordings are explained in
          \cref{sec:preprocessing}.
    \item \textbf{Data multiplication}: Learning systems need lots of data
          to learn internal parameters. If there is not enough data available,
          domain knowledge can be considered to create new artificial data from
          the original data. In the domain of on-line handwriting recognition,
          data can be multiplied by adding rotated variants.
    \item \textbf{Feature extraction}: A feature is high-level information
          derived from the raw data after preprocessing. Some systems like
          Detexify take the result of the preprocessing step, but many compute
          new features. Those features can be designed by a human engineer or
          learned. Non-raw data features have the advantage that less
          training data is needed since the developer uses knowledge about
          handwriting to compute highly discriminative features. Various
          features are explained in \cref{sec:features}.
\end{enumerate}

After these steps, it is a classification task for which the classifier has to
learn internal parameters before it can classify new recordings.We classified
recordings by computing constant-sized feature vectors and using
\glspl{MLP}. There are many ways to adjust \glspl{MLP} (number of neurons and
layers, activation functions) and their training (learning rate, momentum,
error function). Some of them are described in~\cref{sec:mlp-training} and the
evaluation results are presented in \cref{ch:Optimization-of-System-Design}.
%!TEX root = write-math-ba-paper.tex

\section{Data and Implementation}
We used $\num{369}$ symbol classes with a total of $\num{166898}$ labeled
recordings. Each class has at least $\num{50}$ labeled recordings, but over
$200$ symbols have more than $\num{200}$ labeled recordings and over $100$
symbols have more than $500$ labeled recordings.
The data was collected by two crowd-sourcing projects (Detexify and
\href{http://write-math.com}{write-math.com}) where users wrote
symbols, were then given a list ordered by an early classification system and
clicked on the symbol they wrote.

The data of Detexify and \href{http://write-math.com}{write-math.com} was
combined, filtered semi-automatically and can be downloaded via
\href{http://write-math.com/data}{write-math.com/data} as a compressed tar
archive of CSV files.

All of the following preprocessing and feature computation algorithms were
implemented and are publicly available as open-source software in the Python
package \texttt{hwrt}.
%!TEX root = write-math-ba-paper.tex

\section{Algorithms}
\subsection{Preprocessing}\label{sec:preprocessing}
Preprocessing in symbol recognition is done to improve the quality and
expressive power of the data. It makes follow-up tasks like feature extraction
and classification easier, more effective or faster. It does so by resolving
errors in the input data, reducing duplicate information and removing
irrelevant information.

Preprocessing algorithms fall into two groups: Normalization and noise
reduction algorithms.

A very important normalization algorithm in single-symbol recognition is
\textit{scale-and-shift}~\cite{Thoma:2014}. It scales the recording so that
its bounding box fits into a unit square. As the aspect ratio of a recording is
almost never 1:1, only one dimension will fit exactly in the unit square. For
this paper, it was chosen to shift the recording in the direction of its bigger
dimension into the $[0,1] \times [0,1]$ unit square. After that, the recording
is shifted in direction of its smaller dimension such that its bounding box is
centered around zero.

Another normalization preprocessing algorithm is
resampling~\cite{Guyon91,Manke01}. As the data points on the pen trajectory are
generated asynchronously and with different time-resolutions depending on the
used hardware and software, it is desirable to resample the recordings to have
points spread equally in time for every recording. This was done by linear
interpolation of the $(x,t)$ and $(y,t)$ sequences and getting a fixed number
of equally spaced points per stroke.

\textit{Stroke connection} is a noise reduction algorithm which is mentioned
in~\cite{Tappert90}. It happens sometimes that the hardware detects that the
user lifted the pen where the user certainly didn't do so. This can be detected
by measuring the Euclidean distance between the end of one stroke and the
beginning of the next stroke. If this distance is below a threshold, then the
strokes are connected.

Due to a limited resolution of the recording device and due to erratic
handwriting, the pen trajectory might not be smooth. One way to smooth is
calculating a weighted average and replacing points by the weighted average of
their coordinate and their neighbors coordinates. Another way to do smoothing
is to reduce the number of points with the Douglas-Peucker
algorithm to the points that are more relevant for the
overall shape of a stroke and then interpolate the stroke between those points.
The Douglas-Peucker stroke simplification algorithm is usually used in
cartography to simplify the shape of roads. It works recursively to find a
subset of points of a stroke that is simpler and still similar to the original
shape. The algorithm adds the first and the last point $p_1$ and $p_n$ of a
stroke to the simplified set of points $S$. Then it searches the point $p_i$ in
between that has maximum distance from the line $p_1 p_n$. If this distance is
above a threshold $\varepsilon$, the point $p_i$ is added to $S$. Then the
algorithm gets applied to $p_1 p_i$ and $p_i p_n$ recursively. It is described
as \enquote{Algorithm 1} in~\cite{Visvalingam1990}.

\subsection{Features}\label{sec:features}
Features can be \textit{global}, that means calculated for the complete
recording or complete strokes. Other features are calculated for single points
on the pen trajectory and are called \textit{local}.

Global features are the \textit{number of strokes} in a recording, the
\textit{aspect ratio} of a recordings bounding box or the
\textit{ink} being used for a recording. The ink feature gets calculated by
measuring the length of all strokes combined. The re-curvature, which was
introduced in~\cite{Huang06}, is defined as
\[\text{re-curvature}(stroke) := \frac{\text{height}(stroke)}{\text{length}(stroke)}\]
and a stroke-global feature.

The simplest local feature is the coordinate of the point itself. Speed,
curvature and a local small-resolution bitmap around the point, which was
introduced by Manke, Finke and Waibel in~\cite{Manke1995}, are other local
features.

\subsection{Multilayer Perceptrons}\label{sec:mlp-training}
\Glspl{MLP} are explained in detail in~\cite{Mitchell97}. They can have
different numbers of hidden layers, the number of neurons per layer and the
activation functions can be varied. The learning algorithm is parameterized by
the learning rate $\eta \in (0, \infty)$, the momentum $\alpha \in [0, \infty)$
and the number of epochs.

The topology of \glspl{MLP} will be denoted in the following by separating the
number of neurons per layer with colons. For example, the notation
$160{:}500{:}500{:}500{:}369$ means that the input layer gets 160~features,
there are three hidden layers with 500~neurons per layer and one output layer
with 369~neurons.

\glspl{MLP} training can be executed in various different ways, for example
with \acrfull{SLP}. In case of a \gls{MLP} with the topology
$160{:}500{:}500{:}500{:}369$, \gls{SLP} works as follows: At first a \gls{MLP}
with one hidden layer ($160{:}500{:}369$) is trained. Then the output layer is
discarded, a new hidden layer and a new output layer is added and it is trained
again, resulting in a $160{:}500{:}500{:}369$ \gls{MLP}. The output layer is
discarded again, a new hidden layer is added and a new output layer is added
and the training is executed again.

Denoising auto-encoders are another way of pretraining. An
\textit{auto-encoder} is a neural network that is trained to restore its input.
This means the number of input neurons is equal to the number of output
neurons. The weights define an \textit{encoding} of the input that allows
restoring the input. As the neural network finds the encoding by itself, it is
called auto-encoder. If the hidden layer is smaller than the input layer, it
can be used for dimensionality reduction~\cite{Hinton1989}. If only one hidden
layer with linear activation functions is used, then the hidden layer contains
the principal components after training~\cite{Duda2001}.

Denoising auto-encoders are a variant introduced in~\cite{Vincent2008} that
is more robust to partial corruption of the input features. It is trained to
get robust by adding noise to the input features.

There are multiple ways how noise can be added. Gaussian noise and randomly
masking elements with zero are two possibilities.
\cite{Deeplearning-Denoising-AE} describes how such a denoising auto-encoder
with masking noise can be implemented. The corruption $\varkappa \in [0, 1)$ is
the probability of a feature being masked.
%!TEX root = write-math-ba-paper.tex

\section{Optimization of System Design}\label{ch:Optimization-of-System-Design}
In order to evaluate the effect of different preprocessing algorithms, features
and adjustments in the \gls{MLP} training and topology, the following baseline
system was used:

Scale the recording to fit into a unit square while keeping the aspect ratio,
shift it as described in \cref{sec:preprocessing},
resample it with linear interpolation to get 20~points per stroke, spaced
evenly in time. Take the first 4~strokes with 20~points per stroke and
2~coordinates per point as features, resulting in 160~features which is equal
to the number of input neurons. If a recording has less than 4~strokes, the
remaining features were filled with zeroes.

All experiments were evaluated with four baseline systems $B_{hl=i}$, $i \in \Set{1,
2, 3, 4}$, where $i$ is the number of hidden layers as different topologies
could have a severe influence on the effect of new features or preprocessing
steps. Each hidden layer in all evaluated systems has $500$ neurons.

Each \gls{MLP} was trained with a learning rate of $\eta = 0.1$ and a momentum
of $\alpha = 0.1$. The activation function of every neuron in a hidden layer is
the sigmoid function. The neurons in the
output layer use the softmax function. For every experiment, exactly one part
of the baseline systems was changed.


\subsection{Random Weight Initialization}
The neural networks in all experiments got initialized with a small random
weight

\[w_{i,j} \sim U(-4 \cdot \sqrt{\frac{6}{n_l + n_{l+1}}}, 4 \cdot \sqrt{\frac{6}{n_l + n_{l+1}}})\]

where $w_{i,j}$ is the weight between the neurons $i$ and $j$, $l$ is the layer
of neuron $i$, and $n_i$ is the number of neurons in layer $i$. This random
initialization was suggested on
\cite{deeplearningweights} and is done to break symmetry.

This can lead to different error rates for the same systems just because the
initialization was different.

In order to get an impression of the magnitude of the influence on the different
topologies and error rates the baseline models were trained 5 times with
random initializations.
\Cref{table:baseline-systems-random-initializations-summary}
shows a summary of the results. The more hidden layers are used, the more do
the results vary between different random weight initializations.

\begin{table}[h]
    \centering
    \begin{tabular}{crrr|rrr} %chktex 44
    \toprule
    \multirow{3}{*}{System}  & \multicolumn{6}{c}{Classification error}\\
    \cmidrule(l){2-7}
               & \multicolumn{3}{c}{Top-1}   & \multicolumn{3}{c}{Top-3}\\
               & Min                   & Max                   & Mean                  & Min                  & Max                  & Mean\\\midrule
    $B_{hl=1}$ & $\SI{23.1}{\percent}$ & $\SI{23.4}{\percent}$ & $\SI{23.2}{\percent}$ & $\SI{6.7}{\percent}$ & $\SI{6.8}{\percent}$ & $\SI{6.7}{\percent}$ \\
    $B_{hl=2}$ & \underline{$\SI{21.4}{\percent}$} & \underline{$\SI{21.8}{\percent}$}& \underline{$\SI{21.6}{\percent}$} & $\SI{5.7}{\percent}$ & \underline{$\SI{5.8}{\percent}$} & \underline{$\SI{5.7}{\percent}$}\\
    $B_{hl=3}$ & $\SI{21.5}{\percent}$ & $\SI{22.3}{\percent}$ & $\SI{21.9}{\percent}$ & \underline{$\SI{5.5}{\percent}$} & $\SI{5.8}{\percent}$ & \underline{$\SI{5.7}{\percent}$}\\
    $B_{hl=4}$ & $\SI{23.2}{\percent}$ & $\SI{24.8}{\percent}$ & $\SI{23.9}{\percent}$ & $\SI{6.0}{\percent}$ & $\SI{6.4}{\percent}$ & $\SI{6.2}{\percent}$\\
    \bottomrule
    \end{tabular}
    \caption{The systems $B_{hl=1}$ -- $B_{hl=4}$ were randomly initialized,
             trained and evaluated 5~times to estimate the influence of random
             weight initialization.}
\label{table:baseline-systems-random-initializations-summary}
\end{table}


\subsection{Stroke connection}
In order to solve the problem of interrupted strokes, pairs of strokes
can be connected with stroke connection algorithm. The idea is that for
a pair of consecutively drawn strokes $s_{i}, s_{i+1}$ the last point $s_i$ is
close to the first point of $s_{i+1}$ if a stroke was accidentally split
into two strokes.

$\SI{59}{\percent}$ of all stroke pair distances in the collected data are
between $\SI{30}{\pixel}$ and $\SI{150}{\pixel}$. Hence the stroke connection
algorithm was evaluated with $\SI{5}{\pixel}$, $\SI{10}{\pixel}$ and
$\SI{20}{\pixel}$.
All models top-3 error improved with a threshold of $\theta = \SI{10}{\pixel}$
by at least $\num{0.2}$ percentage points, except $B_{hl=4}$ which did not notably
improve.


\subsection{Douglas-Peucker Smoothing}
The Douglas-Peucker algorithm was applied with a threshold of $\varepsilon =
0.05$, $\varepsilon = 0.1$ and $\varepsilon = 0.2$ after scaling and shifting,
but before resampling. The interpolation in the resampling step was done
linearly and with cubic splines in two experiments. The recording was scaled
and shifted again after the interpolation because the bounding box might have
changed.

The result of the application of the Douglas-Peucker smoothing with $\varepsilon
> 0.05$ was a high rise of the top-1 and top-3 error for all models $B_{hl=i}$.
This means that the simplification process removes some relevant information and
does not---as it was expected---remove only noise. For $\varepsilon = 0.05$
with linear interpolation some models top-1 error improved, but the
changes were small. It could be an effect of random weight initialization.
However, cubic spline interpolation made all systems perform more than
$\num{1.7}$ percentage points worse for top-1 and top-3 error.

The lower the value of $\varepsilon$, the less does the recording change after
this preprocessing step. As it was applied after scaling the recording such that
the biggest dimension of the recording (width or height) is $1$, a value of
$\varepsilon = 0.05$ means that a point has to move at least $\SI{5}{\percent}$
of the biggest dimension.


\subsection{Global Features}
Single global features were added one at a time to the baseline systems. Those
features were re-curvature
$\text{re-curvature}(stroke) = \frac{\text{height}(stroke)}{\text{length}(stroke)}$
as described in \cite{Huang06}, the ink feature which is the summed length
of all strokes, the stroke count, the aspect ratio and the stroke center points
for the first four strokes. The stroke center point feature improved the system
$B_{hl=1}$ by $\num{0.3}$~percentage points for the top-3 error and system $B_{hl=3}$ for
the top-1 error by $\num{0.7}$~percentage points, but all other systems and
error measures either got worse or did not improve much.

The other global features did improve the systems $B_{hl=1}$ -- $B_{hl=3}$, but not
$B_{hl=4}$. The highest improvement was achieved with the re-curvature feature. It
improved the systems $B_{hl=1}$ -- $B_{hl=4}$ by more than $\num{0.6}$~percentage points
top-1 error.


\subsection{Data Multiplication}
Data multiplication can be used to make the model invariant to transformations.
However, this idea seems not to work well in the domain of on-line handwritten
mathematical symbols. We tripled the data by adding a version that is rotated
3~degrees to the left and another one that is rotated 3~degrees to the right
around the center of mass. This data multiplication made all classifiers for
most error measures perform worse by more than $\num{2}$~percentage points for
the top-1 error.

The same experiment was executed by rotating by 6~degrees and in another
experiment by 9~degrees, but those performed even worse.

Also multiplying the data by a factor of 5 by adding two 3-degree rotated
variants and two 6-degree rotated variant made the classifier perform worse
by more than $\num{2}$~percentage points.


\subsection{Pretraining}\label{subsec:pretraining-evaluation}
Pretraining is a technique used to improve the training of \glspl{MLP} with
multiple hidden layers.

\Cref{table:pretraining-slp} shows that \gls{SLP} improves the classification
performance by $\num{1.6}$ percentage points for the top-1 error and
$\num{1.0}$ percentage points for the top-3 error. As one can see in
\cref{fig:training-and-test-error-for-different-topologies-pretraining}, this
is not only the case because of the longer training as the test error is
relatively stable after $\num{1000}$ epochs of training. This was confirmed
by an experiment where the baseline systems where trained for $\num{10000}$
epochs and did not perform notably different.

\begin{figure}[htb]
    \centering
    \begin{tikzpicture}
    \begin{axis}[
            axis x line=middle,
            axis y line=middle,
            enlarge y limits=true,
            xmin=0,
            % xmax=1000,
            ymin=0.18, ymax=0.4,
            minor ytick={0, 0.01, ..., 1},
            % width=15cm, height=8cm,     % size of the image
            grid = both,
            minor grid style={dashed, gray!30},
            major grid style={gray!40},,
            %grid style={dashed, gray!30},
            ylabel=error,
            xlabel=epoch,
            legend cell align=left,
            legend style={
                at={(0.5,-0.1)},
                anchor=north,
                legend columns=2
            }
         ]
          \addplot[mark=x,green] table [each nth point=20,x=epoch, y=testerror, col sep=comma] {baseline-1.csv};
          \addplot[mark=x,orange] table [each nth point=20,x=epoch, y=testerror, col sep=comma] {baseline-2.csv};
          \addplot[mark=x,red] table [each nth point=20,x=epoch, y=testerror, col sep=comma] {baseline-2-pretraining.csv};
          \legend{{1 hidden layer},
                  {2 hidden layers},
                  {2 hidden layers with pretraining}}
    \end{axis}
\end{tikzpicture}
    \caption{Training- and test error by number of trained epochs for different
             topologies with \acrfull{SLP}. The plot shows
             that all pretrained systems performed much better than the systems
             without pretraining. All plotted systems did not improve
             with more epochs of training.}
\label{fig:training-and-test-error-for-different-topologies-pretraining}
\end{figure}

\begin{table}[tb]
    \centering
    \begin{tabular}{lrrrr}
    \toprule
    \multirow{2}{*}{System}  & \multicolumn{4}{c}{Classification error}\\
    \cmidrule(l){2-5}
                & Top-1                  & Change               & Top-3                & Change                 \\\midrule
    $B_{hl=1}$     & $\SI{23.2}{\percent}$  & -                    & $\SI{6.7}{\percent}$ & - \\
    $B_{hl=2,SLP}$ & $\SI{19.9}{\percent}$ & $\SI{-1.7}{\percent}$ & $\SI{4.7}{\percent}$ & $\SI{-1.0}{\percent}$\\
    $B_{hl=3,SLP}$ & \underline{$\SI{19.4}{\percent}$} & $\SI{-2.5}{\percent}$ & \underline{$\SI{4.6}{\percent}$} & $\SI{-1.1}{\percent}$\\
    $B_{hl=4,SLP}$ & $\SI{19.6}{\percent}$ & $\SI{-4.3}{\percent}$ & \underline{$\SI{4.6}{\percent}$} & $\SI{-1.6}{\percent}$\\
    \bottomrule
    \end{tabular}
    \caption{Systems with 1--4 hidden layers which used \acrfull{SLP}
             compared to the mean of systems $B_{hl=1}$--$B_{hl=4}$ displayed
             in \cref{table:baseline-systems-random-initializations-summary}
             which used pure gradient descent. The \gls{SLP}
             systems clearly performed worse.}
\label{table:pretraining-slp}
\end{table}


Pretraining with denoising auto-encoder lead to the much worse results listed in
\cref{table:pretraining-denoising-auto-encoder}. The first layer used a $\tanh$
activation function. Every layer was trained for $1000$ epochs and the
\gls{MSE} loss function. A learning-rate of $\eta = 0.001$, a corruption of
$\varkappa = 0.3$ and a $L_2$ regularization of $\lambda = 10^{-4}$ were
chosen. This pretraining setup made all systems with all error measures perform
much worse.

\begin{table}[tb]
    \centering
    \begin{tabular}{lrrrr}
    \toprule
    \multirow{2}{*}{System}  & \multicolumn{4}{c}{Classification error}\\
    \cmidrule(l){2-5}
                 & Top-1                  & Change               & Top-3                & Change                 \\\midrule
    $B_{hl=1,AEP}$ & $\SI{23.8}{\percent}$ & $\SI{+0.6}{\percent}$ & $\SI{7.2}{\percent}$ & $\SI{+0.5}{\percent}$\\
    $B_{hl=2,AEP}$ & \underline{$\SI{22.8}{\percent}$} & $\SI{+1.2}{\percent}$ & $\SI{6.4}{\percent}$ & $\SI{+0.7}{\percent}$\\
    $B_{hl=3,AEP}$ & $\SI{23.1}{\percent}$ & $\SI{+1.2}{\percent}$ & \underline{$\SI{6.1}{\percent}$} & $\SI{+0.4}{\percent}$\\
    $B_{hl=4,AEP}$ & $\SI{25.6}{\percent}$ & $\SI{+1.7}{\percent}$ & $\SI{7.0}{\percent}$ & $\SI{+0.8}{\percent}$\\
    \bottomrule
    \end{tabular}
    \caption{Systems with denoising \acrfull{AEP} compared to pure
             gradient descent. The \gls{AEP} systems performed worse.}
\label{table:pretraining-denoising-auto-encoder}
\end{table}

%!TEX root = write-math-ba-paper.tex

\section{Summary}
Four baseline recognition systems were adjusted in many experiments and their
recognition capabilities were compared in order to build a recognition system
that can recognize 396 mathematical symbols with low error rates as well as to
evaluate which preprocessing steps and features help to improve the recognition
rate.

All recognition systems were trained and evaluated with
$\num{\totalCollectedRecordings{}}$ recordings for \totalClassesAnalyzed{}
symbols. These recordings were collected by two crowdsourcing projects
(\href{http://detexify.kirelabs.org/classify.html}{Detexify} and
\href{write-math.com}{write-math.com}) and created with various devices. While
some recordings were created with standard touch devices such as tablets and
smartphones, others were created with the mouse.

\Glspl{MLP} were used for the classification task. Four baseline systems with
different numbers of hidden layers were used, as the number of hidden layer
influences the capabilities and problems of \glspl{MLP}.

All baseline systems used the same preprocessing queue. The recordings were
scaled and shifted as described in \ref{sec:preprocessing}, resampled with
linear interpolation so that every stroke had exactly 20~points which are
spread equidistant in time. The 80~($x,y$) coordinates of the first 4~strokes
were used to get exactly $160$ input features for every recording. The baseline
system $B_{hl=2}$ has a top-3 error of $\SI{5.7}{\percent}$.

Adding two slightly rotated variants for each recording and hence tripling the
training set made the systems $B_{hl=3}$ and $B_{hl=4}$ perform much worse, but
improved the performance of the smaller systems.

The global features re-curvature, ink, stoke count and aspect ratio improved
the systems $B_{hl=1}$--$B_{hl=3}$, whereas the stroke center point feature
made $B_{hl=2}$ perform worse.

Denoising auto-encoders were evaluated as one way to use pretraining, but by
this the error rate increased notably. However, \acrlong{SLP} improved the
performance decidedly.

The stroke connection algorithm was added to the preprocessing steps of the
baseline system as well as the re-curvature feature, the ink feature, the
number of strokes and the aspect ratio. The training setup of the baseline
system was changed to \acrlong{SLP} and the resulting model was trained with a
lower learning rate again. This optimized recognizer $B_{hl=2,c}'$ had a top-3
error of $\SI{4.0}{\percent}$. This means that the top-3 error dropped by over
$\num{1.7}$ percentage points in comparison to the baseline system $B_{hl=2}$.

A top-3 error of $\SI{4.0}{\percent}$ makes the system usable for symbol
lookup. It could also be used as a starting point for the development of a
multiple-symbol classifier.

The aim of this work was to develop a symbol recognition system which is easy
to use, fast and has high recognition rates as well as evaluating ideas for
single symbol classifiers. Some of those goals were reached. The recognition
system $B_{hl=2,c}'$ evaluates new recordings in a fraction of a second and has
acceptable recognition rates.

% Many algorithms were evaluated. However, there are still many other
% algorithms which could be evaluated and, at the time of this work, the best
% classifier $B_{hl=2,c}'$ is only available through the Python package
% \texttt{hwrt}. It is planned to add an web version of that classifier online.

\section{Optimized Recognizer}
All preprocessing steps and features that were useful were combined to create a
recognizer that performs best.

All models were much better than everything that was tried before. The results
of this experiment show that single-symbol recognition with
\totalClassesAnalyzed{} classes and usual touch devices and the mouse can be
done with a top-1 error rate of $\SI{18.6}{\percent}$ and a top-3 error of
$\SI{4.1}{\percent}$. This was
achieved by a \gls{MLP} with a $167{:}500{:}500{:}\totalClassesAnalyzed{}$ topology.

It used the stroke connection algorithm to connect of which the ends were less
than $\SI{10}{\pixel}$ away, scaled each recording to a unit square and shifted
as described in \ref{sec:preprocessing}. After that, a linear resampling step
was applied to the first 4 strokes to resample them to 20 points each. All
other strokes were discarded.

\goodbreak
The 167 features were\mynobreakpar%
\begin{itemize}
     \item the first 4 strokes with 20 points per stroke resulting in 160
           features,
     \item the re-curvature for the first 4 strokes,
     \item the ink,
     \item the number of strokes and
     \item the aspect ratio of the bounding box
\end{itemize}

\Gls{SLP} was applied with $\num{1000}$ epochs per layer, a
learning rate of $\eta=0.1$ and a momentum of $\alpha=0.1$. After that, the
complete model was trained again for $1000$ epochs with standard mini-batch
gradient descent resulting in systems $B_{hl=1,c}'$ -- $B_{hl=4,c}'$.

After the models $B_{hl=1,c}$ -- $B_{hl=4,c}$ were trained the first $1000$ epochs,
they were trained again for $\num{1000}$ epochs with a learning rate of $\eta =
0.05$. \Cref{table:complex-recognizer-systems-evaluation} shows that
this improved the classifiers again.

\begin{table}[htb]
    \centering
    \begin{tabular}{lrrrr}
    \toprule
    \multirow{2}{*}{System}  & \multicolumn{4}{c}{Classification error}\\
    \cmidrule(l){2-5}
              & Top-1                 & Change                & Top-3                & Change\\\midrule
    $B_{hl=1,c}$ & $\SI{21.0}{\percent}$ & $\SI{-2.2}{\percent}$ & $\SI{5.2}{\percent}$ & $\SI{-1.5}{\percent}$\\
    $B_{hl=2,c}$ & $\SI{18.3}{\percent}$ & $\SI{-3.3}{\percent}$ & $\SI{4.1}{\percent}$ & $\SI{-1.6}{\percent}$\\
    $B_{hl=3,c}$ & \underline{$\SI{18.2}{\percent}$} & $\SI{-3.7}{\percent}$ & \underline{$\SI{4.1}{\percent}$} & $\SI{-1.6}{\percent}$\\
    $B_{hl=4,c}$ & $\SI{18.6}{\percent}$ & $\SI{-5.3}{\percent}$ & $\SI{4.3}{\percent}$ & $\SI{-1.9}{\percent}$\\\midrule
    $B_{hl=1,c}'$ & $\SI{19.3}{\percent}$ & $\SI{-3.9}{\percent}$ & $\SI{4.8}{\percent}$ & $\SI{-1.9}{\percent}$ \\
    $B_{hl=2,c}'$ & \underline{$\SI{17.5}{\percent}$} & $\SI{-4.1}{\percent}$ & \underline{$\SI{4.0}{\percent}$} & $\SI{-1.7}{\percent}$\\
    $B_{hl=3,c}'$ & $\SI{17.7}{\percent}$ & $\SI{-4.2}{\percent}$ & $\SI{4.1}{\percent}$ & $\SI{-1.6}{\percent}$\\
    $B_{hl=4,c}'$ & $\SI{17.8}{\percent}$ & $\SI{-6.1}{\percent}$ & $\SI{4.3}{\percent}$ & $\SI{-1.9}{\percent}$\\
    \bottomrule
    \end{tabular}
    \caption{Error rates of the optimized recognizer systems. The systems
             $B_{hl=i,c}'$ were trained another $\num{1000}$ epochs with a learning rate
             of $\eta=0.05$.}
\label{table:complex-recognizer-systems-evaluation}
\end{table}

%!TEX root = write-math-ba-paper.tex

\section{Evaluation}

The optimized classifier was evaluated on three publicly available data sets:
\verb+MfrDB_Symbols_v1.0+ \cite{Stria2012}, CROHME~2011 \cite{Mouchere2011},
and CROHME~2012 \cite{Mouchere2012}.

\verb+MfrDB_Symbols_v1.0+ contains recordings for 105~symbols, but for
11~symbols less than 50~recordings were available. For this reason, the
optimized classifier was evaluated on 94~of the 105~symbols.

The evaluation results are given in \cref{table:public-eval-results}.

\begin{table}[htb]
    \centering
    \begin{tabular}{lcrr}
    \toprule
    \multirow{2}{*}{Dataset}  & \multirow{2}{*}{Symbols}  & \multicolumn{2}{c}{Classification error}\\
    \cmidrule(l){3-4}
              & & Top-1                 & Top-3                \\\midrule
    MfrDB       & 94 & $\SI{8.4}{\percent}$  & $\SI{1.3}{\percent}$ \\
    CROHME 2011 & 56 & $\SI{10.2}{\percent}$ & $\SI{3.7}{\percent}$ \\
    CROHME 2012 & 75 & $\SI{12.2}{\percent}$ & $\SI{4.1}{\percent}$ \\
    \bottomrule
    \end{tabular}
    \caption{Error rates of the optimized recognizer systems. The systems
             output layer was adjusted to the number of symbols it should
             recognize and trained with the combined data from
             write-math and the training given by the datasets.}
\label{table:public-eval-results}
\end{table}


\bibliographystyle{IEEEtranSA}
\bibliography{write-math-ba-paper}
\end{document}
