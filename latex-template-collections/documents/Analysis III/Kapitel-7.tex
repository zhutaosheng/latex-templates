In diesem Kapitel sei stets \(\emptyset\neq X\in \fb_d\).

\begin{satz}
\label{Satz 7.1}
Sei \(U\in\fb_k, t_0\in U\) und es sei \(f\colon U\times X\to \mdr\) eine Funktion mit:
\begin{enumerate}
	\item 	Für jedes \(t\in U\) ist \(x\mapsto f(t,x)\) messbar.
	\item 	Es existiert eine Nullmenge \(N\subseteq X\) so, dass \(t\mapsto f(t,x)\) für jedes \(x\in X\setminus N\) stetig in $t_0$ ist.
	\item 	Es existiert eine integrierbare Funktion \(g\colon X\to [0,\infty]\) und zu jedem \(t\in U\) existiert eine Nullmenge \(N_t\subseteq X\) so, dass für
		jedes \(t\in U\) und jedes \(x\in X\setminus N_t\) gilt: \[ \lvert f(t,x)\rvert \leq g(x) \]
\end{enumerate}
Dann ist \(x\mapsto f(t,x)\) für jedes \(t\in U\) integrierbar. Ist \(F\colon U\to\mdr\) definiert durch
\[ F(t):=\int_Xf(t,x)\,dx,\]
so ist $F$ stetig in $t_0$.
\end{satz}

Also: \[ \lim\limits_{t\to t_0}\int_X f(t,x)\,dx = \lim\limits_{t\to t_0}F(t)=F(t_0) = \int_X f(t_0,x)\,dx =\int_X\lim\limits_{t\to t_0} f(t,x)\,dx \]

\begin{beweis}
Aus (1) und (3) folgt, dass \(x\mapsto f(t,x)\) für jedes \(t\in U\) integrierbar ist (zur Übung). Sei \((t_n)\) eine Folge in $U$ mit \(t_n\to t_0\) und
\[g_n(x):=f(t_n,x) \ (\natn, x\in X) \]
Setze \[ \tilde N := N\cup \left(\bigcup^\infty_{n=1}N_{t_n} \right) \]
Aus \ref{Lemma 5.1} folgt, dass \(\tilde N\) eine Nullmenge ist. Voraussetzung (2) liefert \(g_n(x)\to f(t_0,x)\) für jedes \(x\in X\setminus\tilde N\), also gilt
\[g_n(x)\to f(t_0,x) \text{ fast überall auf } X\]
Voraussetzung (3) liefert \(\lvert g_n(x)\rvert = \lvert f(t_n,x)\rvert \leq g(x) \) für jedes \(\natn\) und \(x\in X\setminus\tilde N\). Aus \ref{Satz 6.2} folgt
\[ F(t_n) = \int_X f(t_n,x)\,dx = \int_Xg_n\,dx \longrightarrow \int_X f(t_0,x)\,dx = F(t_0) \]
\end{beweis}

\textbf{Bezeichnung}\\
Sei \(I\subseteq\mdr\) ein Intervall, \(a:=\inf I\) und \(b:=\sup I\), wobei \(a=-\infty\) oder \(b=+\infty\) zugelassen sind. Weiter sei \(f\colon I\to\imdr\) integrierbar
(oder $f$ ist messbar und \(\geq 0\)) und
\[\int\limits^b_af(x)\,dx:=\int\limits_{(a,b)}f_{|(a,b)}(x)\,dx \]
Dann ist
\[ \int_I f(x) dx = \int_{(a,b)} f(x) dx\]
Ist z.B. \(I=[a,b)\), dann gilt, da \(\{a\}\) eine Nullmenge ist: \[\int_If\,dx=\int_{\{a\}}f\,dx + \int_{(a,b)}f\,dx= \int_{(a,b)}f\,dx \]

\begin{folgerung}
\label{Folgerung 7.2}
Sei \(I\subseteq\mdr\) ein Intervall, \(a=\inf I\) und \(f\colon I\to\mdr\) sei integrierbar. Definiert man \(F\colon I\to\mdr\) durch
\[F(t):=\int^t_a f(x)\,dx,\] so ist \(F\in C(I)\).
\end{folgerung}

\begin{beweis}
Für \(x,t\in I\) definiere \(h(t,x):=\mathds{1}_{(a,t)}f(x)\). Dann ist \(F(t)=\int_I h(t,x)\,dx\) und
\[\lvert h(t,x)\rvert = \mathds{1}_{(a,t)}\cdot \lvert f(x)\rvert \leq \lvert f(x)\rvert \text{ für alle } t,x\in I\]
Aus \ref{Satz 4.9} folgt, dass \(\lvert f\rvert\) integrierbar ist. Sei \(t_0\in I\) und \(N:=\{t_0\}\), also eine Nullmenge.
Dann ist \(t\mapsto h(t,x)\) für jedes \(x\in I\setminus N\) stetig in \(t_0\) (zur Übung). Die Behauptung folgt aus \ref{Satz 7.1}.
\end{beweis}

\begin{satz}
\label{Satz 7.3}
Sei \(U\subseteq \mdr^k\) offen, \(f\colon U\times X\to\mdr\) eine Funktion. Es sei \(g\colon X\to [0,+\infty]\) integrierbar und \(N\subseteq X\) sei eine Nullmenge.
Weiter gelte:
\begin{enumerate}
	\item 	für jedes \(t\in U\) sei \(x\mapsto f(t,x)\) integrierbar.
	\item 	für jedes \(x\in X\setminus N\) sei \(t\mapsto f(t,x)\) partiell differenzierbar auf $U$.
	\item 	\(\left\lvert \frac{ \partial f}{\partial t_j} \right\rvert \leq g(x) \) für jedes \(x\in X\setminus N\), jedes \(t\in U\) und jedes \(j\in\{1,\dots,k\}\)
\end{enumerate}
Ist dann \(F\colon U\to\mdr\) definiert durch \[F(t):=\int_X f(t,x)dx\] so ist $F$ auf $U$ partiell differenzierbar und für jedes \( t\in U\) sowie jedes \( j\in\{1,\dots,k\}\) gilt:
\[ \frac{\partial F}{\partial t_j}(t) = \int_X\frac{\partial f}{\partial t_j}(t,x)\,dx \]
\end{satz}
\textbf{Also: } \( \frac{\partial}{\partial t_j}\int_X f(t,x)\,dx = \int_X \frac{\partial f}{\partial t_j}(t,x)\,dx \).

\begin{beweis}
Sei o.B.d.A. \(k=1\), also \(U\subseteq\mdr\). Sei \(t_0\in U\) und \((h_n)\) eine Folge mit \(h_n\to 0\) und \(h_n\neq 0\) für alle \(\natn\).
Setze \[g_n(x):=\frac{f(t_0+h_n,x)-f(t_0,x)}{h_n} \ \ (x\in X, \natn) \]
Aus Voraussetzung (2) folgt für jedes \(x\in X\setminus N\): \[ g_n(x)\to \frac{\partial f}{\partial t}(t_0,x) \ \ (n\to\infty) \]
Nach dem Mittelwertsatz aus Analysis 1 existiert für jedes \(x\in X\setminus N\) und jedes \(\natn\) ein \(s_n=s_n(x)\) zwischen \(t_0\) und \(t_0+h_n\) mit:
\[ \left\lvert g_n(x) \right\rvert = \left\lvert \frac{\partial f}{\partial t}(s_n,x)\right\rvert \overset{(3)}\leq g(x) \]
Aus \ref{Satz 6.2} folgt \[ \int_X g_n\,dx \longrightarrow \int_X\frac{\partial f}{\partial t}(t_0,x)\,dx \]
Es ist nach Konstruktion  gerade \(\int_X g_n\,dx =\frac{F(t_0+h_n)-F(t_0)}{h_n} \).
\end{beweis}
