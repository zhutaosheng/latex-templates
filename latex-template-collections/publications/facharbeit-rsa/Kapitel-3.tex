\section{Restklassen}
Teilt man eine ganze Zahl $a$ durch eine ganze Zahl $m \neq 0$, so
bleibt ein Rest $r \in \mathbb{N}_0$. Anhand aller möglichen Reste
$0 \leq r < m$ teilt man nun alle Zahlen in $|m|$ Teilmengen ein.
Diese Teilmengen nennt man Restklassen. Man sagt, alle Zahlen, die
den selben Rest $r$ beim Teilen durch $m$ lassen, gehören der selben
Restklasse modulo $m$ an\footnote{[Forster], S. 45}.
Ein Beispiel aus dem Alltag sind Zeitangaben. Man schreibt nicht 348
Minuten, sondern 5 Stunden und 48 Minuten. Es wird also modulo 60
gerechnet. Auch in der Grund-schule rechnet man mit Restklassen
modulo 10, wenn man ganze Zahlen in Einer, Zehner und Hunderter
unterteilt.\\
Ein weiteres Beispiel ist die Einteilung in  gerade und ungerade
Zahlen. Bleibt bei einer Zahl kein Rest beim Teilen durch zwei, so
wird sie als "`gerade"' bezeichnet und ist in einer Restklasse
modulo 2 mit allen anderen geraden Zahlen.
