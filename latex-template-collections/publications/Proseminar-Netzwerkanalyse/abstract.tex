In dieser Arbeit wird der DYCOS-Algorithmus, wie er in \cite{aggarwal2011}
vorgestellt wurde, erklärt. Er arbeitet auf Graphen, deren Knoten teilweise mit
Beschriftungen versehen sind und ergänzt automatisch Beschriftungen für Knoten,
die bisher noch keine Beschriftung haben. Dieser Vorgang wird
\enquote{Klassifizierung} genannt. Dazu verwendet er die Struktur des Graphen
sowie textuelle Informationen, die den Knoten zugeordnet sind. Die in
\cite{aggarwal2011} beschriebene experimentelle Analyse ergab, dass er auch auf
dynamischen Graphen mit 19\,396 bzw. 806\,635 Knoten, von denen nur
14\,814 bzw. 18\,999 beschriftet waren, innerhalb von weniger als
einer Minute auf einem Kern einer Intel Xeon 2.5\,GHz~CPU mit 32\,G~RAM
ausgeführt werden kann.\\
Zusätzlich wird \cite{aggarwal2011} kritisch Erörtert und und es werden
mögliche Erweiterungen des DYCOS-Algorithmus vorgeschlagen.

\textbf{Keywords:} DYCOS, Label Propagation, Knotenklassifizierung
