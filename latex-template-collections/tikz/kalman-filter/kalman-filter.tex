\documentclass[varwidth=true, border=2pt]{standalone}

\usepackage{pgfplots}
\usepackage{tikz}
\usetikzlibrary{shapes,arrows,arrows.meta}
\pgfplotsset{compat=1.13}
\usepackage{mathtools}
\usepackage{amssymb}

\begin{document}
\tikzstyle{block} = [draw, rectangle, minimum width=6em, align=center,fill=gray!5]
\tikzstyle{arrow} = [-latex, very thick]
\newcommand*{\tran}{\top}

\begin{tikzpicture}[auto, node distance=2cm,>=latex']
    \setlength{\abovedisplayskip}{0pt}

    % Place the blocks
    \node[text width=1cm,
          rounded corners=3pt,
          block,
          label={[above,align=center]{Initial\\State}}] at (-1, 0) (initial)
          {$$\mathbf{x}_0$$  $$P_0$$};
    \node at (0.25, 0.1) (sum) {};
    \node[block, text width=6cm,
          label={[above,align=center]{Prediction}}] at (4, 0) (prediction)
          {\begin{align*}
            \mathbf{x}_{k+1}^{(P)} &= A \mathbf{x}_k + B {\color{orange} a_k}\\
            P_{k+1}^{(P)} &= A P_k A^\tran + C_k^{(r_s)}
           \end{align*}};
    \node [block, right of=prediction,
            node distance=3cm, text width=1.4cm] at (6, -2) (iterUpdate)
            {$$k \leftarrow k + 1$$};
    \node [block, text width=6cm,
           label={[above,align=center]{Innovation}}] at (4, -4) (innovation)
           {\begin{align*}
              K_k &= P_k^{(P)} H^\tran {\left (H P_k^{(P)} H^\tran + C_k^{(r_m)} \right)}^{-1}\\
              {\color{blue} \mathbf{x}_k} &= (I - K_k H) \mathbf{x}_k^{(P)} + K_k {\color{orange} z_k}\\
              {\color{blue} P_k} &= (I - K_k H) P_k^{(P)}
            \end{align*}};

    % Connect the nodes
    \draw [arrow] (initial) -- (prediction);
    \draw [arrow] (prediction.east) -| (iterUpdate.north);
    \draw [arrow] (iterUpdate) |- (innovation);
    \draw [arrow] (innovation.west) -|  (sum);
\end{tikzpicture}

\end{document}
